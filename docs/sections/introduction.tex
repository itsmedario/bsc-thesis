\chapter{Introduction}

\section{Background and Motivation}

Over the course of the past years, computer science has become significantly more important not only in the academic world, but also in all of modern society. In Switzerland, the introduction of Lehrplan 21 \cite{Lehrplan21} accelerated this process. In order to cope with these changes, the Center for Computer Science Education at ETH Zurich (also known as ABZ \cite{ABZ}) has set itself the target to support the introduction of computer science as one of the foundation subjects in the obligatory school curricula. 

ABZ has been publishing educational material such as textbooks, courses and online platforms. The purpose is to bring computer science into classrooms such that basic computer science concepts may be familiarized among teen pupil. In particular, the disciples will be playfully introduced into topics such as information and data handling, encryption, the finding of strategies and application of algorithms.


\section{Objectives}

The main objective of the thesis is the design, implementation, and testing of tasks in an online learning environment. Content-wise, the tasks will consist of problems and activities proposed in chapter II “Probleme lösen” of the ABZ textbook “einfach Informatik 3/4” that covers the solving of algorithmic problems. In particular, the simple algorithmic concepts of this chapter will be modelled as tasks that are suitable for elementary school. This newly created learning environment will provide a smooth introduction to latter algorithms using a bottom-up approach such that difficulty increases as the users progress.

The platform is to be deployed in schools and utilized alongside the corresponding ABZ textbook. Thus, it is imperative that this web-based environment is stable and works fluently on common operating systems. Also, it should be suitable for teachers as well as their disciples by providing tasks complementary to the other learning material. The environment will then be embedded into the already existing learning repository of ABZ.

As part of the Bachelor’s Thesis, the student is expected to tackle the problem by solving
three sub-tasks, according to the Bachelor's thesis proposal:

\subsection{Design and Concept of the Tasks and the Environment}
\label{subsection:t1}
For a start, an exhaustive study of the relevant chapters of the new ABZ textbook “einfach Informatik 3/4” needs to be done in order to thoroughly understand the algorithmic concepts and how they can be modelled. Also, the corresponding didactic publications of Chair of Information Technology and Education as well as further literature on teaching methods in computer science should be studied in a way that its concepts can be taken into consideration. Then, the tasks that should be implemented are to be fixed and modelled, and their user interface is to be designed. In addition,
different frameworks should be compared and the details of the implementation need to be fixed.

\subsection{Implementation and Testing of the Environment}
\label{subsection:t2}
As a next step, the before-hand modelled tasks will be implemented. In order to do this, the technical programming concepts and languages have to be examined and understood. Furthermore, the code to be implemented should be consistent with the above-mentioned design and concept. The environment will finally be tested under various circumstances and a report is made.

\subsection{Presentation of the Results}
\label{subsection:t3}
At the end, the procedure will be documented in a comprehensive report. Not only the thesis but also the environment will be explained and demonstrated in a concluding presentation.

\newpage
\section{Thesis Outline}
This report will describe the implementation project in detail, starting with its design choices and consequences thereof. Then, the general architecture of the environment will be presented, including details of the program user interface. In the latter section, important components for the build-up of the environment will be discussed. Next, the general task concept will be laid out. Then, for each task set, some remarkable implementation features will be focused on more closely. Before drawing a conclusion, this report also features the results and the experience of live testing of the learning environment in different age groups.


%The source code, live testing results, this report and all other used scripts and documents can be found on the ETH GitLab at \url{https://gitlab.ethz.ch/dnaepfer/bsc-thesis}.