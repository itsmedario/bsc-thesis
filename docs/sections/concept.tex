\chapter{Task Concept}
\label{chapter:concept}

\section{Introduction}
\label{section:introduction}
The main objective of information technology is to automatize the search for solutions if tasks are assigned. Here, this concept will be applied to algorithmic problems. The latter term describes a collection of tasks of high similarity, also called problem instances in this thesis. Due to high similarity, such task collections can be solved with a specific individual algorithm. An algorithm is a universal method consisting of precise rules describing how to solve any problem instance of the corresponding problem set.

In this chapter, the general idea behind the task sets will be presented. This includes the didactic idea underneath the task levels, as well the design of the different variations in each task set.

\section{Difficulty Levels}
\label{section:levels}

The main goal that underlies the learning environment is to equip students to be able to solve problems on their own. According to Bloom's Taxonomy, there are different levels of complexity depending on the cognitive dimensions of a task. The levels of each task set are designed accordingly: Starting with simple challenges called classification tasks, the tasks are designed such that the difficulty of the assignments improves gradually. In this way, they are introduced step by step into the algorithmic concepts that should be acquired. Active learning and self-improvement is promoted. The students should learn to think in a way that enables them to approach new challenges and assignments with sophisticated steps and to solve them.

The tasks are designed such that the different levels of a task set look very similar. This creates a recognition value, meaning that the user has already been familiarized with the problem and thus knows the general concept of the task. 

With the possibility of multiple permissible solutions for most problem instances, the student's creativity is required and promoted. The students learn to solve similar problems with yet different solutions.

With random distributions that are included in a majority of tasks, the students are required to solve the task on their own since there is no solution or hint from an external source like the neighbour's screen.


\section{Level 1 - Verify Solutions}
\label{section:verify}
The first difficulty level tests the student's capability to interpret a given problem set together with a provided proposition for a solution to the problem. The proposition then has to be categorized as either a correct or incorrect proposition. A proposition is permissible if and only if it fully complies to the task specifications. The tasks are modelled as decision problems where the user has to answer at least one question regarding the correctness of the proposition. The student is required to interpret the proposition of the problem instance correctly, fully understand the criteria, and then evaluate the given solution proposition accordingly. Furthermore, in some cases, the students learn to justify why a specific problem instance is or is not permissible.

\section{Level 2 - Find Solutions}
\label{section:find}
At this difficulty level, the user is confronted with a task for which he or she has to find a permissible solution on his or her own. The learning objective is to take the matter into one's hand and actively find ways to deal with a new challenge. This is more difficult than just verifying solutions, since the student's creativity to come up with an own solution is required. The only condition a solution must fulfil is that the proposed solution has to be valid, the quality of the solution is not taken into consideration on this level. All solutions are considered of equal value.
For one exercise type, there is an advanced version of this level: The  \nameref{chapter:weights} exercise has another level that instructs the user to find a solution by him- or herself as well. In this case, however, the computer has already taken several steps towards a permissible solution proposition, so the amount of permissible solutions decreases a lot. This makes the exercise more difficult to solve than the regular level 2 exercise.

\newpage
\section{Level 3 - Optimize Solutions}
\label{section:optimize}
The highest difficulty level requires from the user to compare different solution propositions according to given criteria. The user has to compare the quality of these propositions and then decide for the optimal solution. Here, it is imperative that the possible solutions are strictly limited. The variety of the solutions must be small enough that the student is able to list all the different solutions and can then decide which one is the optimal proposition.